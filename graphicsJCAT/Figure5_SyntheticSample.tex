\documentclass[journal]{IEEEtran}

\usepackage[T1]{fontenc}% optional T1 font encoding

\usepackage[pdftex]{graphicx}

\ifCLASSOPTIONcompsoc
\usepackage[caption=false,font=normalsize,labelfon
t=sf,textfont=sf]{subfig}
\else
\usepackage[caption=false,font=footnotesize]{subfi
g}
\fi

\begin{document}

\begin{figure*}
\centering
\subfloat[Sample from the synthetic dataset.]{\includegraphics[width=0.2\textwidth]{ syn_obj_slice.png}
\label{fig:syn_obj_slice}}
\hfil
\subfloat[Encoded shape prior associated to Fig. \ref{fig:syn_obj_slice}.]{\includegraphics[width=0.2\textwidth]{ syn_cad_slice.png}
\label{fig:syn_cad_slice}}
\hfil
\subfloat[Sinogram corresponding to Fig. \ref{fig:syn_obj_slice}.]{\includegraphics[width=0.2\textwidth]{ syn_obj_sin.png}
\label{fig:syn_obj_sin}}
\hfil
\subfloat[Sinogram corresponding to Fig. \ref{fig:syn_cad_slice}.]{\includegraphics[width=0.2\textwidth]{ syn_cad_sin.png}
\label{fig:syn_cad_sin}}
\caption{Sample from the volume of the synthetic dataset. (\ref{fig:syn_obj_slice}) shows the ground-truth reconstruction and (\ref{fig:syn_cad_slice}) shows a representation of the shape priors showing each object with a different randomly assigned gray value, in which no plastic container is visible. In an attempt to reproduce noise in the image, a Gaussian noise is added to the material of uniform densities. The sinograms shown in (\ref{fig:syn_obj_sin}) and (\ref{fig:syn_cad_sin}) are obtained using the forward radon transform with a parallel beam geometry.}
        \label{fig:Sample synthetic}
\end{figure*}          

\end{document}

