\documentclass[journal]{IEEEtran}

\usepackage[T1]{fontenc}% optional T1 font encoding

\usepackage{cite}
\usepackage[pdftex]{graphicx}
\usepackage{amsmath,amssymb,amsfonts}
\interdisplaylinepenalty=2500
\usepackage{array} 
\usepackage{epstopdf}
\usepackage{multirow}
\usepackage{subfiles}

\newcommand{\xr}{X-Ray }
\newcommand{\xrs}{X-Rays }

\ifCLASSOPTIONcompsoc
\usepackage[caption=false,font=normalsize,labelfon
t=sf,textfont=sf]{subfig}
\else
\usepackage[caption=false,font=footnotesize]{subfi
g}
\fi

\begin{document}

\title{Sinogram Enhancement with Generative Adversarial Networks using Shape Priors}

\author{Emilien~Valat,
        Katayoun~Farrahi,
        and~Thomas~Blumensath% <-this % stops a space
\thanks{Emilien V., Katayoun F. and Thomas B. are with the University of Southampton, Southampton, United Kingdom}
\thanks{This work is supported by the DSTL/DGA PhD scheme.}
}
\markboth{Journal of Computer Assisted Tomography}%
{Shell \MakeLowercase{\textit{et al.}}: \xr Computed Tomography Reconstruction with Generative Adversarial Networks using Shape Priors}
\maketitle

\begin{abstract}
Compensating scarce measurements by inferring them from computational models is a way to address ill-posed inverse problems. We tackle Limited Angle Tomography by completing the set of acquisitions using a generative model and prior-knowledge about the scanned object.
Using a Generative Adversarial Network as model and Computer-Assisted Design data as shape prior, we demonstrate a quantitative and qualitative advantage of our technique over other state-of-the-art methods. Inferring a substantial number of consecutive missing measurements, we offer an alternative to other image inpainting techniques that fall short of providing a satisfying answer to our research question: can X-Ray exposition be reduced by using generative models to infer lacking measurements?  
\end{abstract}

\begin{IEEEkeywords}
Generative Adversarial Networks, Image Inpainting with Edge Information, \xr Computed Tomography, Computer Assisted Design Data, Shape Priors
\end{IEEEkeywords}

\IEEEpeerreviewmaketitle

 \section{Introduction}

\subfile{sections/introduction}

\section{Proposed Approach}

\subfile{sections/proposedApproach}

\section{Datasets}

\subfile{sections/Datasets}

\section{Implementation Details}

\subfile{sections/implementationDetails}

\section{Results}

\subfile{sections/results}

\section{Discussion and Conclusion}

\subfile{sections/conclusion}

\section{Bibliography}

\subfile{sections/bibliography}

\section{Images}

\subfile{sections/images}

\end{document}