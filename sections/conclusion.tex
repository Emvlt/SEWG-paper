\documentclass[../main.tex]{subfiles}
\graphicspath{{\subfix{../images/}}}
\begin{document}
\label{sec:conclusion}
One must note that the problem tackled by the Unet-only approach is not exactly the same as ours, as the limited-angle tomography is addressed by removing acquisition at regular angular intervals and training on patches of the sinogram. This method is not designed for inferring acquisitions when a substantial number of them are missing. Indeed, if the patch size is 50 and the number of consecutive missing acquisitions is more than 50, which is only 7\% missing acquisitions in the experimental set-up of the paper, the method cannot work as no true information would be given to the network. We also believe that it was not designed for a use case where all the sinogram was processed at once, even if \cite{lee2018deep} mentions that larger patch size had no issues to be dealt with. Our comment on the GAN-only approach is that the sinograms that are of interest are much smaller in size than the ones we are interested in: 128*128 and 180*180 instead of 256*256 in our case. Their method is directly inspired by \cite{yeh2016semantic}, where results are achieved on 64*64 images and where learning is based on contextual as well as discriminator losses. We think that the GAN inpainting without CAD prior fails at the task because we could not train it satisfyingly using the same procedure as we followed for our architecture. 

In this paper we have shown how to use a GAN to exploit shape prior information to address the image inpainting with edge information problem in XCT. Our experiments demonstrate the significant advantage that our method offers over SOTA methods that do not include shape prior information, for inferring a large number of consecutive missing acquisitions. We also demonstrated that the CAD prior facilitates the GAN training. The main limitation of our approach is the failure at extrapolating faults unexpected by the shape priors. As discussed in \ref{inconsistencies}, we believe that it is due to the fact that all missing acquisitions are generated at once. Also, the image reconstruction process is not studied here but we think that weighting the ground-truth and generated acquisitions could improve the final image quality. We conclude by reflecting on the method used for sinogram enhancement. The term inpainting refers to images and we believe that there is a misuse of this technique for sinograms. Indeed, a sinogram is a sequence of acquisitions that is concatenated into an image for visualisation purposes but it is not an image as the word  is intended for a photograph. We then believe that even if it is a valid way to address the problem of inferring missing acquisitions, it is under-performing. Indeed, no information about the sampling geometry is used, nor about the position of the acquisition to infer. This method also constrains the sinograms as they have to be relatively small, two-dimensional, and have the same number of acquisitions as there are pixel detectors. In a future study, we will focus on inferring one acquisition at a time, accounting for the angular proximity to other acquisitions and specificity of the interpolation of acquisitions in XCT.

\end{document}