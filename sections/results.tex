\documentclass[../main.tex]{subfiles}
\graphicspath{{\subfix{../images/}}}
\begin{document}
\label{sec:results}
Even if the choice of evaluation metrics for synthetic images remains an open problem \cite{salimans2016improved}, the evolution of well-known loss functions is a good estimate of the effect of the network on the sinogram. Thus, we chose to report the PSNR and the SSIM between predictions and ground-truth. We compare our method to state of the art methods from the literature. In particular, we compare it to the sinogram inpainting method of \cite{yoo2019sinogram} and the U-net approach in \cite{lee2018deep}, where we also implemented a version that used additional CAD constraints. These constraints were enforced here by concatenating the scarce sinogram, the CAD sinogram and the mask before training the U-net with it.

\subsection{Estimation Performance on the SophiaBeads Dataset}
We have two volumes of 200 slices each. Each volume is split into 90 percent training set and 10 percent testing set. Training is performed on the training set of the volume reconstructed with 1024 projections and testing is done on the testing set of the volume reconstructed with 512 projections.

\subsubsection{Impact on the sinogram}
\label{impact_on_the_sinogram_sophiabeads}

As can be seen in Table\ref{table:results_sophiaBeads_sinograms}, the pix2pix architecture combined with the CAD prior outperforms all other compared approaches. It is interesting to see that using the CAD prior significantly enhances the performance of the U-net. It is also worth mentioning that the scaling operation slightly improves the quality of the CAD replacement. The inpainting process has limitations that we  will detail in \ref{sec:conclusion}.

\subsubsection{Impact on the reconstructions}
The reconstructions are made using the Simultaneous Iterative Reconstruction Technique algorithm and implemented with TIGRE for Python \cite{biguri2016tigre}.

As can be seen in Table\ref{table:results_sophiaBeads_sinograms}, not only does the pix2pix architecture combined with the CAD prior outperform all other approaches considered in the image space, but it is the only one that yields an improvement compared to the reconstruction made from the scarce sinogram. It is also important to underline that methods performing well in the sinogram space do not necessarily yield an improvement in the image space. This might be due to the reconstruction algorithm used. Fig. \ref{fig:reconstructions} shows the reconstructions associated with each method for 128 missing acquisitions.

\subsection{Effect of Inconsistencies Between the Two Modalities on the Synthetic Dataset}
\label{inconsistencies}
The advantage of our method here is less strong and a qualitative inspection shows that the network fails at inferring the hole pattern. It shows a limited ability to recover defects. In Fig. \ref{fig:images_with_holes}, we show four examples of the difference between the image reconstructed with our method and the target image, for 60 percent missing acquisitions. This indicates that the GAN fails to learn the fine structure of the object's attenuation.

One reason for this might be that we are estimating a very large number of missing projections, which is an extremely challenging task. Also, the nature of holes makes the choice of loss function delicate, as they are "dents" in an acquisition and the GAN produces a high-frequency signal, which is very similar to this pattern. In a future study, we plan to modify the architecture to generate one acquisition at a time, hoping to overcome this limitation. One must also note that the image reconstruction process does not weigh real versus GAN-inferred acquisitions differently, which might improve performance as it would take uncertainty in estimates into account.
\end{document}