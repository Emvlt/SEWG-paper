\documentclass[../main.tex]{subfiles}
\graphicspath{{\subfix{../images/}}}
\begin{document}

\IEEEPARstart{X}{-Ray} Computed Tomography (XCT) is a versatile 3D imaging technique that allows the estimation of volumetric \xr attenuation profiles. It produces cross-sectional images of bodies sensitive to this radiation by sampling an object from different viewing angles to reconstruct an image from this sequence of acquisitions. Yet, \xrs are toxic for in-vivo diagnosis and time-consuming in industrial testing. There is a trade-off between sufficient sampling for high-quality images and \xr intake for time and health constraints. Can computational methods exploit prior knowledge about the scanned object to compensate scarce acquisitions by inferring measurements?

We use Generative Adversarial Networks (GAN) to complete the sequence of scarce acquisitions by inferring them from Computer-Assisted Design data. When imaging a slice through an object from the viewing angle $\theta$, with a detector with $R$ pixels, the measurement can be described as
\begin{equation*}
F(\theta,r) = \int_{x} \int_{y}  f(x, y) \delta(x \cos(\theta) + y \sin(\theta) -r )  \ dx \ dy
\end{equation*}
with $r$ the index of pixels on the detector, $(x,y)$ the coordinates of the object and $f$ the density function of the object, that maps a spatial position to a material density. Let $\Theta$ be the set of viewing angles at which the object is sampled. The set of measurements
\begin{equation*}
	S=\{F(\theta,r)\} \textrm{ for all } \theta \in \Theta
\end{equation*} 
is the sinogram of the image $f(x, y)$. It can be represented as an image of size $\vert \Theta \vert \times R$. As such, missing acquisition in the sequence is represented as zero-valued pixels along one dimension of the sinogram. The problem of inferring missing acquisitions from a scarce sinogram is then similar to the one of inferring arbitrarily large regions in images based on image semantics, known as semantic inpainting.

Semantic image inpainting is a constrained image generation problem \cite{pathak2016context}. Missing parts of an image are inpainted using a generative network and solving an optimisation problem. GAN \cite{goodfellow2014generative} and their fully-convolutional version Deep-Convolutional GAN (DCGAN) \cite{radford2015unsupervised} are adapted to this task: they were used for image \cite{yeh2016semantic} and sinogram \cite{yoo2019sinogram} inpainting. The optimisation relies on finding the "closest" encoding in the latent space of the GAN distribution by minimising a penalty function that encompasses contextual and conditional information. This method suffers from several limitations:
\begin{itemize}
	\item Walking the latent space of the distribution can only yield certain improvements in the image generated by the GAN: \cite{jahanian2019steerability} shows that images can only be transformed to some degree (brightness, zoom, rotation). Not only is the transformation corresponding to inpainting is not defined, but it has no certainty to be achievable by "steering" the generated output, especially when guided by a generic loss function and not by a supervised walk.
	\item The optimisation function adds computational time and hyperparameters. In addition to training the generative model, the optimisation process is time-consuming and also requires fine-tuning of the learning rate and number of iterations. 
\end{itemize}
As an alternative to this process, we use CAD data as a prior and train a Unet-GAN \cite{ronneberger2015u}, \cite{isola2017image} to infer missing parts of the sinogram given shape information about the scanned object.

Shape information is often available in both medical and industrial imaging, but is rarely used in XCT reconstruction. For instance, in medical imaging, projects such as \cite{ackerman1998visible} and \cite{xu2007boundary} demonstrate the potential of using numerical shape models of a generic human body to minimise \xr exposure. An alternative is to extract prior information from earlier scans \cite{huang2013iterative, abbas2013super}. For manufactured components, Computer Assisted Design (CAD) drawings are often available, providing strong constraints on object shape. These types of priors provide estimates of object boundary locations, though might not contain information on exact \xr absorption within an object, nor do they contain information about unknown defects and inclusions. 

In this paper, we minimise \xr intake by inpainting the scarce sinogram with a GAN and a shape prior. The main advantage compared to other inpainting methods is the side-stepping of the optimisation process. Unlike the other image inpainting methods, we focus on inpainting the missing part of the sinogram only, reducing the complexity of the task. Our main findings are:
\begin{itemize}
	\item Exploiting the specificity of XCT data enhances the CAD prior. Before feeding the CAD to the GAN, we rescale its values so that they match what has been observed in the scarce sinogram.
	\item Prior information about the shape of the object improves significantly the quality of state-of-the-art (SOTA) sinogram-enhancing techniques. We show that including the CAD prior to other methods that address a similar problem yields a significant improvement of their performance.
	\item Not only does our method enhance the sinogram in terms of Peak Signal-to-Noise Ratio (PSNR) and Structural SIMilarity (SSIM)compared to other SOTA techniques, we also report an improvement of the reconstructed image quality.
\end{itemize}
\end{document}