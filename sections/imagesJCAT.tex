\documentclass[../main.tex]{subfiles}
\graphicspath{{\subfix{../images/}}}
\begin{document}

\begin{figure*}
\centering
{\includegraphics[width=0.15\textwidth]{example-image-a}}
\caption{Visual explanation of the inpainting method.}
\label{fig:inpainting_method_explanation}
\end{figure*}

\begin{figure*}
\centering
{\includegraphics[width=0.15\textwidth]{example-image-a}}
\caption{Visual explanation of the training of the GAN.}
\label{fig:training_procedure_explanation}
\end{figure*}

\begin{figure*}
\centering
\subfloat[Slice of the SophiaBeads volume.]{\includegraphics[width=0.2\textwidth]{example-image-a}
\label{fig:sophiabeads_obj_slice}}
\hfil
\subfloat[Shape prior associated to Fig. \ref{fig:sophiabeads_obj_slice}.]{\includegraphics[width=0.2\textwidth]{example-image-a}
\label{fig:sophiabeads_cad_slice}}
\hfil
\subfloat[Sinogram corresponding to Fig. \ref{fig:sophiabeads_obj_slice}.]{\includegraphics[width=0.2\textwidth]{example-image-a}
\label{fig:sophiabeads_obj_sin}}
\hfil
\subfloat[Sinogram corresponding to Fig. \ref{fig:sophiabeads_cad_slice}.]{\includegraphics[width=0.2\textwidth]{example-image-a}
\label{fig:sophiabeads_cad_sin}}
\caption{Sample from the volume of the SophiaBeads dataset. Fig \ref{fig:sophiabeads_obj_slice} shows the reconstructed acquisition and Fig \ref{fig:sophiabeads_cad_slice} shows a representation of shape priors showing each object with a different randomly assigned gray value, in which no plastic container is visible. The sinograms shown in Fig. \ref{fig:sophiabeads_obj_sin} and Fig. \ref{fig:sophiabeads_cad_sin} are obtained using the forward radon transform with a parallel beam geometry from \ref{fig:sophiabeads_obj_slice} and \ref{fig:sophiabeads_cad_slice}, respectively.}
\label{fig:Sample SophiaBeads}
\end{figure*}

\begin{figure*}
\centering
\subfloat[Sinogram with missing acquisitions drawn from the SophiaBeads dataset.]{\includegraphics[width=0.2\textwidth]{example-image-a}
\label{fig:sophia_sin_example}}
\hspace{5mm}%
\subfloat[Mask associated with Fig. \ref{fig:sophia_sin_example}.]{\includegraphics[width=0.2\textwidth]{example-image-a}
\label{fig:sophia_sin_mask}}
\caption{Fig. \ref{fig:sophia_sin_example} shows a sinogram with missing acquisitions and Fig. \ref{fig:sophia_sin_mask} its associated mask.}
\label{fig:sinogram_example}
\end{figure*}
    
\begin{figure*}
\centering
\subfloat[Sample from the synthetic dataset.]{\includegraphics[width=0.2\textwidth]{example-image-a}
\label{fig:syn_obj_slice}}
\hfil
\subfloat[Encoded shape prior associated to Fig. \ref{fig:syn_obj_slice}.]{\includegraphics[width=0.2\textwidth]{example-image-a}
\label{fig:syn_cad_slice}}
\hfil
\subfloat[Sinogram corresponding to Fig. \ref{fig:syn_obj_slice}.]{\includegraphics[width=0.2\textwidth]{example-image-a}
\label{fig:syn_obj_sin}}
\hfil
\subfloat[Sinogram corresponding to Fig. \ref{fig:syn_cad_slice}.]{\includegraphics[width=0.2\textwidth]{example-image-a}
\label{fig:syn_cad_sin}}
\caption{Sample from the volume of the synthetic dataset. (\ref{fig:syn_obj_slice}) shows the ground-truth reconstruction and (\ref{fig:syn_cad_slice}) shows a representation of the shape priors showing each object with a different randomly assigned gray value, in which no plastic container is visible. In an attempt to reproduce noise in the image, a Gaussian noise is added to the material of uniform densities. The sinograms shown in (\ref{fig:syn_obj_sin}) and (\ref{fig:syn_cad_sin}) are obtained using the forward radon transform with a parallel beam geometry.}
        \label{fig:Sample synthetic}
\end{figure*}          

\begin{figure*}
\centering
\subfloat[Example of a sample of the synthetic dataset when internal defects are present.]{\includegraphics[width=0.2\textwidth]{example-image-a}
\label{fig:holes_examples_}}
\caption{Objects with internal defects not encoded in the shape prior.}
\label{fig:holes_examples}
\end{figure*}

\begin{figure*}
\centering
\subfloat[Target reconstruction with 256 acquisitions]{\includegraphics[width=0.2\textwidth]{example-image-a}
\label{fig:reconstructions128_target}}
\hfill
\subfloat[Reconstruction without any interpolation method , i.e from 128 acquisitions]{\includegraphics[width=0.2\textwidth]{example-image-a}
\label{fig:reconstructions128_noInterpolation}}
\hfill
\subfloat[Reconstruction with linear interpolation]{\includegraphics[width=0.2\textwidth]{example-image-a}
\label{fig:reconstructions128_linearInterpolation}}
\hfill
\subfloat[Reconstruction with missing acquisitions replaced by CAD-expected ones]{\includegraphics[width=0.2\textwidth]{example-image-a}
\label{fig:reconstructions128_inferenceCad}}
\hfill
\subfloat[Reconstruction with acquisitions inferred from the Unet without the CAD prior]{\includegraphics[width=0.2\textwidth]{example-image-a}
\label{fig:reconstructions128_NoCadUnet}}
\hfill
\subfloat[Reconstruction with acquisitions inferred from the Unet with the CAD prior]{\includegraphics[width=0.2\textwidth]{example-image-a}
\label{fig:reconstructions128_CadUnet}}
\hfill
\subfloat[Reconstruction with sinogram inpainted by the GAN.]{\includegraphics[width=0.2\textwidth]{example-image-a}
\label{fig:reconstructions128_NoCadGan}}
\hfill
\subfloat[Reconstruction with missing acquisitions inferred from the pix2pix architecture using CAD prior.]{\includegraphics[width=0.2\textwidth]{example-image-a}
\label{fig:reconstructions128_CadGan}}

\label{fig:reconstructions}
\caption{Reconstructions of the first sample of the SophiaBeads test dataset.(\ref{fig:reconstructions128_target}) is the target image, reconstructed from all 256 acquisitions and (\ref{fig:reconstructions128_noInterpolation}) is the image reconstructed from the scarce sinogram. (\ref{fig:reconstructions128_linearInterpolation}) to (\ref{fig:reconstructions128_CadGan}) are the images reconstructed from sinograms enhanced with various methods.}
\end{figure*}  

\begin{figure*}
\centering
\subfloat[Target reconstruction.]{\includegraphics[width=0.15\textwidth]{example-image-a}
\label{fig:holes_tar}}
\hfill
\subfloat[Reconstruction using our method.]{\includegraphics[width=0.15\textwidth]{example-image-a}
\label{fig:holes_gan}}
\hfill
\subfloat[Difference between the image reconstructed with our method and the target reconstruction.]{\includegraphics[width=0.15\textwidth]{example-image-a}
\label{fig:holes_gan_tar}}
\hfill
\subfloat[Reconstruction using the shape prior.]{\includegraphics[width=0.15\textwidth]{example-image-a}
\label{fig:holes_cad}}
\hfill
\subfloat[Difference between the image reconstructed with the shape prior and the target reconstruction.]{\includegraphics[width=0.15\textwidth]{example-image-a}
\label{fig:holes_cad_tar}}
\caption{Understanding the action of the GAN on objects with internal defects not expected by the shape prior. (\ref{fig:holes_tar}) shows the target reconstruction, (\ref{fig:holes_gan}) shows the reconstruction with our method and (\ref{fig:holes_gan_tar}) shows the difference between (\ref{fig:holes_gan}) and (\ref{fig:holes_tar}). (\ref{fig:holes_cad}) shows the reconstruction with the shape prior and (\ref{fig:holes_cad_tar}) shows the difference between (\ref{fig:holes_cad}) and (\ref{fig:holes_tar}).}
\label{fig:images_with_holes}
        
\end{figure*}  

\end{document}